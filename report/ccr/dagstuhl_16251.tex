\documentclass{sig-alternate-05-2015}

\pagestyle{plain}

\usepackage{color, colortbl}
\usepackage{multirow}
\usepackage{times}
\usepackage{transparent}
\usepackage{balance}
\usepackage{url}
\usepackage{graphicx}
\usepackage{paralist}
\usepackage{amssymb, framed, graphics}
\usepackage{bytefield}
\usepackage{todonotes}
\usepackage{comment}
\usepackage{algorithm}
\usepackage{algpseudocode}
\usepackage[group-separator={,}]{siunitx}
\usepackage[underline=true]{pgf-umlsd}
\usepackage{hyphenat}
\usepackage{subfigure}
\usepackage{amsmath}
\usepackage{cite}
\usepackage{fancybox,fancyvrb}
\usepackage{url}
\usepackage{fancyhdr}
\usepackage{framed}
\usepackage{xspace}
\usepackage{subfigure}
\usepackage{tabularx}

\pagestyle{fancy}

\renewcommand{\algorithmicrequire}{\textbf{Input:}}
\renewcommand{\algorithmicensure}{\textbf{Output:}}

\newtheorem{theorem}{Theorem}

\newtheorem{definition}{Definition}
\newtheorem{lemma}{Lemma}

\makeatletter
\renewcommand{\paragraph}[1]{\vspace*{0.03in}\noindent{\bf #1.}\hspace{0.25ex \@plus1ex \@minus.2ex}}
\makeatother

\fancypagestyle{plain}{%
   \fancyhf{} %
   \renewcommand{\headrulewidth}{0pt}%
   \fancyfoot[L]{ACM SIGCOMM Computer Communication Review }%
   \fancyfoot[R]{Volume 47 Issue 1, April 2016}%
}
\pagestyle{plain}
\def\refname{}

\usepackage{etex}
\usepackage{etoolbox}

\makeatletter
\patchcmd{\@afterheading}%
    {\clubpenalty \@M}{\clubpenalties 3 \@M \@M 0}{}{}
\patchcmd{\@afterheading}%
    {\clubpenalty \@clubpenalty}{\clubpenalties 2 \@clubpenalty 0}{}{}
\makeatother

%\def\sharedaffiliation{%
%\end{tabular}
%\begin{tabular}{c}}

\begin{document}

\title{Can We Make a Cake and Eat it Too? \\ A Discussion of ICN Security and Privacy} % at Dagstuhl}

\numberofauthors{6}
\author{
\alignauthor
Edith Ngai \\
\affaddr{Uppsala University, SE}\\
\email{edith.ngai@it.uu.se}
\and
\alignauthor
B\"{o}rje Ohlman \\
\affaddr{Ericsson Research, SE}\\
\email{Borje.Ohlman@ericsson.com }
\and
\alignauthor
Gene Tsudik \\
\affaddr{University of California Irvine}\\
\email{gts@ics.uci.edu}
\and
\alignauthor
Ersin Uzun \\
\affaddr{Xerox PARC}\\
\email{ersin.uzun@parc.com}
\and
\alignauthor
Matthias W\"{a}hlisch \\
\affaddr{Freie University Berlin, DE}\\
\email{m.waehlisch@fu-berlin.de}
\and
\alignauthor
Christopher A. Wood \\
\affaddr{University of California Irvine}\\
\email{woodc1@uci.edu}
\end{tabular}
\begin{tabular}{c}
\begin{minipage}[t]{0.9\textwidth}%
\begin{center}
\vspace{11pt} {\normalsize This article is an editorial note submitted to CCR. It
has NOT been peer reviewed. The author takes full responsibility for this
article's technical content. Comments can be posted through CCR Online.}
\end{center}
\end{minipage}\tabularnewline
}

\maketitle

\begin{abstract}
In recent years, Information-centric Networking (ICN) has received much attention from
both academic and industry participants. ICN offers data-centric inter-networking
that is radically different from today's host-based IP networks. Security and
privacy features on today's Internet were originally not present and have been
incrementally retrofitted over the last 35 years. As such, these issues
have become increasingly important as ICN technology gradually matures towards
real-world deployment. Thus, while ICN-based architectures
(e.g., NDN, CCNx, etc.) are still evolving, it is both timely and important to explore
ICN security and privacy issues as well as devise and assess possible mitigation techniques.

This report documents the highlights and outcomes of the Dagstuhl Seminar 16251
on ``Information-centric Networking and Security.'' The goal of which was to bring together
researchers to discuss and address security and privacy issues particular to ICN-based
architectures. Upon finishing the three-day workshop, the outlook of ICN is still
unclear. Many unsolved and ill-addressed problems remain, such as namespace
and identity management, object security and forward secrecy, and privacy. Regardless
of the fate of ICN, one thing is certain: much more research and practical experience
with these systems is needed to make progress towards solving these arduous problems.

% Attendees represented diverse areas of expertise, including: networking,
% security, privacy, software engineering, and formal methods. Through presentations and
% focused working groups, attendees identified and discussed issues relevant to security
% and privacy, and charted paths for their mitigation and future work.
\end{abstract}

\begin{CCSXML}
<ccs2012>
<concept>
<concept_id>10002978.10003006</concept_id>
<concept_desc>Security and privacy~Systems security</concept_desc>
<concept_significance>500</concept_significance>
</concept>
<concept>
<concept_id>10002978.10003014</concept_id>
<concept_desc>Security and privacy~Network security</concept_desc>
<concept_significance>500</concept_significance>
</concept>
<concept>
<concept_id>10003033.10003034</concept_id>
<concept_desc>Networks~Network architectures</concept_desc>
<concept_significance>500</concept_significance>
</concept>
</ccs2012>
\end{CCSXML}

\ccsdesc[500]{Security and privacy~Systems security}
\ccsdesc[500]{Security and privacy~Network security}
\ccsdesc[500]{Networks~Network architectures}

\printccsdesc
\keywords{Information-Centric Networking; Security and Privacy}


\section{Introduction}
Dagstuhl Seminar 16251 on ``Information-centric Networking and Security'' was a three-day
workshop held on June 19-21, 2016 in Dagstuhl, Germany. The goal was to bring together researchers with
different areas of expertise relevant to Information-Centric Networking (ICN) to discuss
security and privacy issues particular to ICN-based architectures. These problems have
become increasingly important as ICN technology gradually matures and gets close to real-world deployment.

\paragraph{Brief History of Dagstuhl Seminars on ICN}
This seminar is the fourth retreat of the ICN community at Schloss Dagstuhl, which
usually meets every two years at this place. In 2010, the term ICN was
coined to assemble different informa\-tion-centric networking approaches,
such as NetInf (4WARD and SAIL projects) and PSIRP in Europe and Content-Centric Networking (CCN) in the US \cite{dagstuhl2010}.
In 2012, reality checks were discussed \cite{dagstuhl2012}. In 2014, special focus was given to scalability
and deployment issues \cite{dagstuhl2014}. And finally, at long last, the most
recent seminar discussed primarily security aspects of these architectures.

\paragraph{Current Research Challenges}
Many ICN-based architectures have the luxury of starting with a clean slate.
As a consequence, threat models in ICN are often distinct from traditional IP-based
networks ~\cite{wsv-bdpts-13,gtuz-ddndn-13}. Differentiating factors between the two include
new application design patterns, trust models and management, as well as a strong
emphasis on object-based, instead of channel-based, security. As ICN develops, it is both
timely and important to explore ICN security and privacy issues in order to devise and
assess possible mitigation techniques. This was the general purpose of the Dagstuhl
seminar. To that end, the seminar focused on the following issues:
%
\begin{compactitem}
\item What are the relevant threat models that ICN must be concerned? How are
they different from those in the IP-based networks?
\item To what extent is trust management a solved problem in ICN? Have we adequately
identified the core elements of a trust model, e.g., with Named-Data Networking (NDN) trust schemas?
\item How practical and realistic is object-based security when framed in the
context of accepted, best-practice privacy measures used in IP-based networks?
\item Are there any new types of cryptographic schemes or primitives that ICN architectures
could use to (a) enable more efficient or secure packet
processing or (b) build a more secure architecture?
\end{compactitem}
%
The seminar answered (entirely or partially) some of these questions and fueled discussions
for others. In the remainder of this document we will review the major themes of the
seminar, discuss many open problems that exist in ICN architectures, and summarize
their outlook going forward.

\section{Driving Themes} \label{sec:themes}
There were several driving themes throughout the course of the seminar, including:
ICN-related threat models, namespace and identity management, privacy, the
so-called ``locator and identifier split,'' ICN and IoT, and future design directions.
In this section we summarize the discussions along each theme in the context
of general ICN-based architectures. References to specific architectures, such as
NDN or CCN, are included where necessary.

\subsection{IP Parity and Threat Models}
ICN attempts to diverge from IP with respect
to the central abstraction of hosts and point-to-point communication between
them. The ICN emphasis on named data and object security instead channel security
is one clear differentiating factor contributing to this divergence. To quantify
the degree by which secret is improved (or worsened), threat models are needed.
In general, they must capture particular design challenges in ICN, such as
infrastructure protection, user-friendly key distribution and trust management, and
content protection and access control.  Given the wide gap between IP and ICN,
there is a tremendous need for standard and well-scrutinized threat models to use in the design phase.

To give an example of a threat that is unique to ICN, consider the
problem of consumer anonymity. By design, the subject and content
of a ICN packets are not the facets to be considered in the context of privacy.
The origin and destination details (e.g., geographical location or position within a network
topology), as well as identity information (e.g., consumer identifying information),
can sometimes harm the network users. As shown by \cite{compagno2015violating}, ICN caching and
interest-collapse mechanisms make ICN itself inherently vulnerable to the possibility for an
adversary to locate consumers. Therefore, the threat model must consider these vulnerabilities
and adversaries capable of exploiting them successfully.

\subsection{Namespace and Identity Management}
In ICN, there is an intimate relationship between trust, identity, and namespace management.
Furthermore, resource naming, which in the current Internet is primarily an application-layer
concern, now directly affects the network layer as well.
%@@@ Edith: The following paragraph looks a bit abstract at the beginning. What do trust preferences and policies mean here? Are we talking about data or name authentication? @@@
In an ideal ICN architecture, applications should be able to express their trust preferences
(using policies) and let some ``middleware'' enforce them throughout the network. This raises
two important questions: (1) what is the minimum set of policies that can be factored out of
all trust models, and (2) what is the middleware that does this enforcement? The trust schema concept
pioneered by the NDN architecture \cite{schemas} is a prime example of a set of rules that can be
used to express most trust models. Among other things, they specify what keys are allowed to
sign what data. Since both keys and data are named resources in NDN and other ICN architectures,
this means that a schema allows for arbitrary hierarchical trust models. It remains to be seen
if other non-hierarchical trust models will be so effortlessly realized in ICN-based architectures.

With respect to (2), it is clear that the network should only be responsible for
validating at most one signature per packet and doing nothing else to enforce
trust models. (Further details about this limitation are provided in \cite{trust}.)
However, despite this limitation, network layer ``trust enforcement'' should not prohibit or prevent other
application-layer trust models. This means that the network functionality must be simpler than
that which is supported by the middleware. Functionalities such as certificate chain
resolution or key retrieval should not be implemented in the core network.
This behavior must be handled by other network nodes (e.g., consumers and producers)
or other middleware entities.
% It is worth noting that end systems are supersets of routers in terms of functionality,
% and thus may route data as well.

Even with an agreement about how and to what extent the network aids in trust
enforcement, we are left with the following major question: how are names registered and managed in ICN?
Namespace ownership is intrinsically tied to an identity. Thus, namespace advertisements under
different namespaces or in different networks must be authenticated with respect to the
claimed owner's identity. In this context, an identity is a public and private key pair.
The community struggled with issues about namespace scale and the practicality of a global
namespace. Questions such as, ``how do NATs work in a global namespace?,'' drove the
discussion. No consensus or common understanding about how namespaces and identities
should be managed was reached. This is still an area of active research.

\subsection{Privacy}
Privacy is and has always been an elusive property in ICN-based architectures.
In many designs, there is a significant amount of information leaked by packets, including
content payloads, signatures, and even the names. One major focus for this seminar
was on privacy with respect to names. In this context, we defined name privacy to be the property
that a so-called ``network name,'' i.e., the name encoded in a packet, has no correlation
or connection with the corresponding content object. Specifically, name privacy means
that a network name reveals nothing about the data inside the content object. Ideally,
names should reveal no more than what is currently revealed by an IP address and port.

Adding name privacy to ICN-based architectures is no easy task. As a thought exercise,
consider how this would work if it were done cleanly, i.e., without some upper-layer
service. To restrict the design space, one might make the following assumptions:
%
\begin{itemize}
\item Content may be requested by an identifier (ID) such as its cryptographic hash digest.
Moreover, revealing the content ID does not compromise privacy.
\item Consumers know the public key of the producer with which they want to communicate.
\item Network names have an implicit routable prefix and application-specific suffix.
By default, consumers do not know the locator and identifier split in a name.
\item Requests may specify the ID of (1) a signature verification key or (2) the expected content.
\end{itemize}
%
Under these assumptions, now consider how a consumer might fetch content.
There are fundamentally two ways to issue a request: (1) with and (2) without a
content ID. In case (1), a request name needs to only contain a routable prefix that will
move the request to some cache or producer which can return the corresponding content.
These locators can be completely separate from the desired content and, therefore, this
approach can satisfy our name privacy goal. However, without implicit knowledge about
the locator for some desired content, an upper-layer service is necessary to obtain said information.

In case (2), the application-specific suffix of a name must not reveal anything about
the data. To achieve this, it must be encrypted. Name encryption introduces a number
of other questions, such as how to obtain the routable prefix, what key to use for
encryption, and how to ``protect'' the result. Assume that the routable prefix is
known and that the producer public key is used for name suffix encryption. If the
resultant content payload is not encrypted then one may be able to infer the name
from its contents. Therefore, the content response itself must also be encrypted.
This requires requests to carry a consumer-generated key that is protected in a
CCA-secure envelope. Otherwise, eavesdroppers could replay requests with the same
encrypted name but their own key to obtain a decrypted response.

In all cases, it seems that to achieve name privacy then one needs some upper-layer
service. Whether its role is to provide the routable prefix for a name, encrypt the
response, or to separate a content ID and locator via some other means is an orthogonal
issue to be resolved. Also, one critical observation is that name privacy seems to, in most cases, invalidate the utility of
shared caches, which puts it at odds with the primary feature of many ICN-based
architectures. Thus, it seems as though name privacy is a property that must be
abandoned in pure name-based ICN designs.

% Thus, our conjecture is that name privacy is not possible natively in the network.

\subsection{Locator and Identifier Split}
The notion of locators and how to fetch data with non-topological names (or
even topological names that are cached off-path) was another major theme of
the seminar. Routing should, possibly, only concern
itself with topological names or addresses. Finding data (objects) with non-topological
names should not be done in the data plane. It should be done via a service.

In CCN, this service could resolve a named root manifest to then resolve locator names by hash.
In NDN, it resolves the link routing hints to allow off-path interest forwarding. In TagNet \cite{papalini2015tagnet},
there is a distinction between Locator names and Descriptor names. Locator names have a
strong binding between their name and a point of attachment. Descriptor (hash) names, on the
other hand, are free-form and could be present anywhere.  One resolves a tag query (of either
type) to a topological locator and then does data transfer on that locator.

This lack of uniformity led to one major question: should locators and identifiers
be split, as in TagNet, or should they be combined? For example, in CCN, if there is a clear locator
field and then a clear identifier tuple (Name, [KeyID Restriction], [Hash Restriction]),
one would get full matching expressiveness with the functionality of nameless object locators.
A similar approach could be done in NDN, though with a different tuple.  There was no
consensus on this idea, though it is worth exploring.

There was also some discussion on the benefit of ICN if one still needs to do an external
name to address lookup. Specifically, is it worth it if one still needs a DNS-like function?
One partial answer is that in the non-global routing space (i.e., data center, maybe IoT, some internal
applications, etc.), one could inject all names into the internal routing protocol and realize
the full benefit of application-specific names.
% Another argument is that it improves
% how one writes software to not have to deal with IP addresses and host-based networking.
% One could also see benefits from a re-designed network stack beyond sockets.

\subsection{ICN and IoT}
The Internet of Things (IoT) is connecting billions of smart devices (e.g., sensors) and is growing
very fast. In the future, users may communicate directly to the sensors, or indirectly through the cloud or gateways.
Considering the communication patterns and future trends, there may be benefits in applying ICN to the IoT. For instance,
the ICN routers connecting to sensors can cache the sensor data to improve the performance of data
dissemination. In addition, the users may obtain sensor data directly from the sensors or from a
nearby ICN router, without going through the cloud. Although ICN for IoT may provide in-network
caching and flexibility in data dissemination, it raises several security concerns:

\begin{itemize}
\item How can the sensors be securely configured at the time of initialization?
\item How can software updates be performed securely?
\item How can we handle mobility in ICN for IoT? For example, each sensor may need a unique publisher
identity, which may change with its location. How does mobility affect naming of data and scalability of routing?
\end{itemize}

There are also several advantages and concerns of caching data at the sensors. Firstly,
sensors are resource-limited devices, which may not have sufficient memory for caching the data.
However, memory resources in the sensors may increase and the
price may decrease in the future. Secondly, it is advantageous to retrieve data directly from the
sensors in certain use cases. For example, it is more efficient to control home lighting without
going through the cloud. Thirdly, when applying cryptography on  the sensors, the encryption time
could be long and cause additional delay in data retrieval. Lastly, the sensors have to be always
on to listen to the interest packets in ICN, which may consume a large amount of energy.
Mitigating this problem might require us to more carefully use scheduling or adaptive duty cycles.

Based on these observations, there are several major outstanding questions. Firstly, what are
some critical use cases for ICN in the IoT and how can we experiment with them to
better understand the application communication patterns and the related security requirements.
Secondly, beyond network stack simplicity, what are other ways in which the IoT might
benefit from ICN? Thirdly, what is the best way to configure and bootstrap the sensors securely?
And finally, what is the cost of providing security for IoT data with ICN? Answers to
some of these questions are already topics of active research \cite{shang2016named}.

\subsection{Access Control}
Another challenge unique to ICN is how to develop scalable object-based
access control mechanisms. A variety of encryption techniques have been used
in the past to protect access to confidential network data \cite{tourani2016security}.
Many design approaches, particularly in CCN and NDN, exist well above the network
layer \cite{Smetters2010,Misra2013,Ion2013,Wood2014,ifip15,yu2015name,ghali2015interest}.
In contrast, publish-subscribe ICNs such as ENCODERS \cite{raykova2015decentralized} integrate
access control into the network layer. It uses multi-authority attribute-based encryption \cite{chase2007multi}
to allow content access to be scoped to selected nodes in the system. Since the
system is completely decentralized, peers serve as brokers that match content from
the publishers with interests expressed by the subscribers. In order to perform such a match,
an intermediate node must be authorized to see both the relevant content tags and
the subscriber interests. In this case,
access control policies applied to the metadata (content tags and subscriber interests)
effectively create reachability constraints that are independent from the one defined by
the routing protocols. Consequently, when performed at the network layer,
this security-routing interaction must be treated carefully during policy definition.

\subsection{Design Exploration}
There are many infrastructure security and privacy problems in IP-based networks that we could try to remedy in ICN-based architectures.
Can ICNs use the NSEC3 strategy of \emph{authenticated denial} \cite{blacka2008dns} to limit incoming requests
for non-existent content as a way of deterring sophisticated distributed DoS (DDoS) attacks?
Can we levereage recent advances in deep packet inspection over encryption data \cite{sherry2015blindbox}
to let forwarders blindly route packets without seeing the content or sensitive information,
e.g., application names? Can we leverage information-theoretic PIR techniques \cite{pir-icn} to support
truly private content queries? Can we prevent correlation
of static content across multiple consumers by adopting randomizable encryption schemes \cite{blazy2011signatures}?
And if so, how can we do so while maintaining the integrity of content? There are many
features that could improve beyond what is done in IP-based networks. Moreover, there are cryptographic
algorithms, schemes, and protocols that could allow us to realize these features. However,
it is unclear whether or not these more esoteric cryptographic schemes can and should be applied in
the network layer of future networks.

\section{Outlook}
XXX


\section{Conclusion}
This paper described in detail the discussions and outcomes of the Dagstuhl 16251
seminar on ICN security and privacy. Despite significant research over the past
half decade, there are still many open problems with solutions that are difficult
to be completely realized with the existing architectures. Are we too invested in
the current architectures to make significant design changes to solve these problems?
Is there something to be gained by sacrificing properties such as privacy in favor
of features such as object security? If so, is this the right tradeoff to make today?
Only future research and development will tell.

\subsection*{Acknowledgements}
We thank Schloss Dagstuhl for providing a stimulating setting for this seminar. Much
progress was made over the course of the seminar and since its completion. This is mainly
because of the ease of face-to-face collaboration and interaction at Dagstuhl.

\begin{comment}
Furthermore, we thank all participants for fruitful \emph{and} open discussions
on very urgent challenges. The following people attend Dagstuhl Seminar 16251:
    Bengt Ahlgren (Swedish Institute of Computer Science, SE),
    Tohru Asami (University of Tokyo, JP),
    Roland Bless (Karlsruher Institut f\"ur Technologie, DE),
    Randy Bush (Internet Initiative Japan Inc., Tokyo, JP),
    Kenneth L. Calvert (University of Kentucky, Lexington, US),
    Antonio Carzaniga (University of Lugano, CH),
    Mauro Conti (University of Padova, IT),
    Lars Eggert (NetApp Deutschland GmbH, Kirchheim, DE),
    Darleen Fisher (NSF, Arlington, US),
    Ashish Gehani (SRI, Menlo Park, US),
    Jussi Kangasharju (University of Helsinki, FI),
    Ghassan Kar\-ame (NEC Laboratories Europe, Heidelberg, DE),
    Dirk Kut\-scher (NEC Laboratories Europe, Heidelberg, DE),
    John Mattsson (Ericsson Research, Stockholm, SE),
    Marc Mosko (Xerox PARC, Palo Alto, US),
    Edith Ngai (Uppsala University, SE),
    B\"orje Ohlman (Ericsson Research, Stockholm, SE),
    J\"org Ott (TU M\"unchen, DE),
    Craig Partridge (BBN Technologies, Cambridge, US),
    Fabio Pianese (Bell Labs, Nozay, FR),
    Sanjiva Prasad (Indian Inst. of Technology, Dehli, IN),
    Thomas C. Schmidt (HAW Hamburg, Hamburg, DE),
    Sebastian Sch�nberg (Intel, Santa Clara, US),
    Christoph Schuba (Ericsson, San Jose, US),
    Glenn Scott (Xerox PARC, Palo Alto, US),
    Jan Seedorf (NEC Laboratories Europe, Heidelberg \& Hochschule f\"ur Technik Stuttgart, DE),
    Tim Strayer (BBN Technologies, Cambridge, US),
    Christian Tschudin (Universit\"at Basel, CH),
    Gene Tsudik (University of California, Irvine, US),
    Ersin Uzun (Xerox PARC, Palo Alto, US),
    Matthias W\"ahlisch (FU Berlin, DE),
    Cedric Westphal (Huawei Technologies, Santa Clara, US),
    Christopher A. Wood (University of California, Irvine, US).
\end{comment}

%\small
\balance
\bibliographystyle{IEEEtran}
\bibliography{IEEEabrv,references}

\end{document}

\documentclass{sig-alternate-10pt}

\pagestyle{plain}

\usepackage{color, colortbl}
\usepackage{multirow}
\usepackage{times}
\usepackage{transparent}
\usepackage{balance}
\usepackage{url}
\usepackage{graphicx}
\usepackage{paralist}
\usepackage{amssymb, framed, graphics}
\usepackage{bytefield}
\usepackage{todonotes}
\usepackage{algorithm}
\usepackage{algpseudocode}
\usepackage[group-separator={,}]{siunitx}
\usepackage[underline=true]{pgf-umlsd}
\usepackage{hyphenat}
\usepackage{subfigure}
\usepackage{amsmath}
\usepackage{cite}
\usepackage{fancybox,fancyvrb}
\usepackage{url}
\usepackage{framed}
\usepackage{xspace}
\usepackage{subfigure}

\renewcommand{\algorithmicrequire}{\textbf{Input:}}
\renewcommand{\algorithmicensure}{\textbf{Output:}}

\newtheorem{theorem}{Theorem}

\newtheorem{definition}{Definition}
\newtheorem{lemma}{Lemma}

\makeatletter
\renewcommand{\paragraph}[1]{\vspace*{0.03in}\noindent{\bf #1.}\hspace{0.25ex \@plus1ex \@minus.2ex}}
\makeatother

\begin{document}
\title{A Thing}

\numberofauthors{1}
\author{
\alignauthor
Peoples \\
\affaddr{Places}\\
\email{Things}
}

\maketitle

\begin{abstract}
In recent years, Information-centric Networking (ICN) has received much attention from
both academic and industry participants. ICN offers a data-centric means of inter-networking
that is radically different from today's host-based IP networks. Security and privacy issues
in ICN have become increasingly important as ICN technology gradually matures and nears
real-world deployment. As is well known, in today's Internet, security and privacy features
were originally not present and had to be incrementally and individually retrofitted
(with varying success) over the last 35 years. In contrast, since ICN-based architectures
(e.g., NDN, CCNx, etc.) are still evolving, it is both timely and important to explore
ICN security and privacy issues as well as devise and assess possible mitigation techniques.

This report documents the program and outcomes of the Dagstuhl Seminar 16251
``Information-centric Networking and Security.'' The goal was to bring together
researchers to discuss and address security and privacy issues particular to ICN-based
architectures. Attendees represented diverse areas of expertise, including: networking,
security, privacy, software engineering, and formal methods. Through presentations and
focused working groups, attendees identified and discussed issues relevant to security
and privacy, and charted paths for their mitigation.
\end{abstract}

\section{Introduction}
Dagstuhl seminar 16251 ``Information-centric Networking and Security'' was a short
workshop held June 19-21, 2016. The goal was to bring together researchers with
different areas of expertise relevant to ICN to discuss security and privacy issues
particular to ICN-based architectures. These problems have become increasingly
important as ICN technology gradually matures and nears real-world deployment.

Threat models are distinct from IP. Differentiating factors between the two include
new application design patterns, trust models and management, as well as a strong
emphasis on object-based, instead of channel-based, security. Therefore, it is both
timely and important to explore ICN security and privacy issues as well as devise and
assess possible mitigation techniques. This was the general purpose of the Dagstuhl
seminar. To that end, the attendees focused on the following issues:
%
\begin{itemize}
\item What are the relevant threat models with which ICN must be concerned? How are
they different from those in IP-based networks?
\item To what extent is trust management a solved problem in ICN? Have we adequately
identified the core elements of a trust model, e.g., with NDN trust schemas?
\item How practical and realistic is object-based security when framed in the
context of accepted privacy measures used in IP-based networks?
\item Are there new types of cryptographic schemes or primitives ICN architectures
should be using or following that will enable (a) more efficient or secure packet
processing or (b) an improved security architecture?
\end{itemize}
%
The seminar answered (entirely or partially) some of these questions and fueled discussions
for others. To begin, all participants briefly introduced themselves. This was followed
by several talks on various topics, ranging from trust management and identity to privacy
and anonymity. Subsequently, the attendees split into working groups to focus more
intensely on specific topics. Working group topics included routing on encrypted names,
ICN and IoT, non-privacy-centric aspects of ICN security, as well as trust and identity in ICN.
Once the working group sessions were over, a representative from each presented outcomes
to all attendees. (These are documented in the remainder of this report.) The major
takeaways from the seminar were as follows.

{\bf First}, the ICN community still does not have a clear answer for how to handle
namespace and identity management. While trust management in ICN can be distributed and
function without a global PKI, it seems difficult to break away from this model for
namespace management and arbitration. This has strong implications on how names are
propagated in the routing fabric. Can any producer application advertise any name,
anywhere in the network? If not, how can name prefix advertisements be constrained or limited?

{\bf Second}, given that ICN focuses on object security, the need for and use of transport
protocols that provide forward secrecy should be deferred to higher layers. Attendees
found that while most ICN-based architectures do not preclude forward secrecy, it should
not be a requirement at the network layer.

{\bf Third}, there is still deep uncertainty about whether ICN should embrace a content
locator and identifier split. Names in architectures such as NDN and CCN serve as both a
locator and identifier of data, though there are extensions that permit explicit locators
(e.g., through the use of NDN LINK objects). This distinction is necessary under the
common understanding that routing should concern itself with topological names. Finding
data through non-topological names should not be in the data plane as part of the global
routing space. However, if we revert to a distinction between topological locators and
identifiers, then features unique to ICN become much more limited. One facet that is
certainly unique to ICN is how software is written. Specifically, we have the opportunity
to move beyond the mental model of a fixed address space and re-design existing network
stacks and APIs.

{\bf Fourth}, privacy seems difficult to achieve without major architectural changes to
ICN-based systems. In particular, since data names reveal a great deal of information to
the passive eavesdropper, privacy demands that names and payloads have no correlation.
However, achieving this seems infeasible without the presence of an upper-layer service
akin to one that would resolve non-topological identifiers to topological names.

{\bf Lastly}, there are no compelling reasons to apply esoteric (and often untested)
cryptographic techniques in ICN, at least at the network layer. Computationally bounded
and ``boring'' cryptographic primitives, such as digital signatures, hash functions, etc.,
should be the extent of per-packet cryptographic processing done by routers. Anything
more would become fodder for Denial-of-Service attacks that could render the entire
infrastructure ineffective. However, architecture designs should not restrict themselves
to specific algorithms. In other words, there must be flexibility in accommodating
multiple (and evolving) cryptographic primitives. This could be useful if, for example,
post-quantum digital signature schemes become necessary for the longevity of content authenticators.

We thank Schloss Dagstuhl for providing a stimulating setting for this seminar. Much
progress was made over the course of the seminar and since its completion. This is mainly
because of the ease of face-to-face collaboration and interaction at Dagstuhl.

\section{Namespace and Identity Management}
XXX

\section{Object Security but No Forward Secrecy}
XXX

\section{Locators and Identifiers}
XXX

\section{The Futility of Privacy}
XXX

\section{Boring Crypto}
XXX

\section{Conclusion and Future Work}
XXX

\small
\bibliographystyle{IEEEtran}
\bibliography{IEEEabrv,references}

\end{document}

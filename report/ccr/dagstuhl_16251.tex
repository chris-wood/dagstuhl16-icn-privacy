\documentclass{sig-alternate-10pt}

\pagestyle{plain}

\usepackage{color, colortbl}
\usepackage{multirow}
\usepackage{times}
\usepackage{transparent}
\usepackage{balance}
\usepackage{url}
\usepackage{graphicx}
\usepackage{paralist}
\usepackage{amssymb, framed, graphics}
\usepackage{bytefield}
\usepackage{todonotes}
\usepackage{comment}
\usepackage{algorithm}
\usepackage{algpseudocode}
\usepackage[group-separator={,}]{siunitx}
\usepackage[underline=true]{pgf-umlsd}
\usepackage{hyphenat}
\usepackage{subfigure}
\usepackage{amsmath}
\usepackage{cite}
\usepackage{fancybox,fancyvrb}
\usepackage{url}
\usepackage{fancyhdr}
\usepackage{framed}
\usepackage{xspace}
\usepackage{subfigure}
\usepackage{tabularx}

\pagestyle{fancy}

\renewcommand{\algorithmicrequire}{\textbf{Input:}}
\renewcommand{\algorithmicensure}{\textbf{Output:}}

\newtheorem{theorem}{Theorem}

\newtheorem{definition}{Definition}
\newtheorem{lemma}{Lemma}

\makeatletter
\renewcommand{\paragraph}[1]{\vspace*{0.03in}\noindent{\bf #1.}\hspace{0.25ex \@plus1ex \@minus.2ex}}
\makeatother

\fancypagestyle{plain}{%
   \fancyhf{} %
   \renewcommand{\headrulewidth}{0pt}%
   \fancyfoot[L]{ACM SIGCOMM Computer Communication Review }%
   \fancyfoot[R]{Volume XX Issue YY, MONTH YEAR}%
}
\pagestyle{plain}
\def\refname{}

%\def\sharedaffiliation{%
%\end{tabular}
%\begin{tabular}{c}}

\begin{document}

\title{Can We Make a Cake and Eat it Too? \\ A Discussion of ICN Security and Privacy at Dagstuhl}

\numberofauthors{6}
\author{
\alignauthor
Edith Ngai \\
\affaddr{Uppsala University, SE}\\
\email{edith.ngai@it.uu.se}
\and
\alignauthor
B\"orje Ohlman \\
\affaddr{Ericsson Research, Stockholm, SE}\\
\email{Borje.Ohlman@ericsson.com }
\and
\alignauthor
Gene Tsudik \\
\affaddr{University of California Irvine}\\
\email{gts@ics.uci.edu}
\and
\alignauthor
Ersin Uzun \\
\affaddr{Xerox PARC}\\
\email{ersin.uzun@parc.com}
\and
\alignauthor
Matthias W\"ahlisch \\
\affaddr{Freie University Berlin, DE}\\
\email{m.waehlisch@fu-berlin.de}
\and
\alignauthor
Christopher A. Wood \\
\affaddr{University of California Irvine}\\
\email{woodc1@uci.edu}
\end{tabular}
\begin{tabular}{c}
\begin{minipage}[t]{0.9\textwidth}%
\begin{center}
\vspace{11pt} {\normalsize This article is an editorial note submitted to CCR. It
has NOT been peer reviewed. The author takes full responsibility for this
article's technical content. Comments can be posted through CCR Online.}
\end{center}
\end{minipage}\tabularnewline
}

\maketitle

\begin{abstract}
In recent years, Information-centric Networking (ICN) has received much attention from
both academic and industry participants. ICN offers a data-centric inter-networking
that is radically different from today's host-based IP networks. Security and privacy issues
in ICN have become increasingly important as ICN technology gradually matures for
real-world deployment. Security and privacy features on today's Internet
were originally not present and have been incrementally retrofitted
over the last 35 years. ICN-based architectures
(e.g., NDN, CCNx, etc.) are still evolving, it is both timely and important to explore
ICN security and privacy issues as well as devise and assess possible mitigation techniques.

This report documents the program and outcomes of the Dagstuhl Seminar 16251
on ``Information-centric Networking and Security.'' The goal was to bring together
researchers to discuss and address security and privacy issues particular to ICN-based
architectures. Attendees represented diverse areas of expertise, including: networking,
security, privacy, software engineering, and formal methods. Through presentations and
focused working groups, attendees identified and discussed issues relevant to security
and privacy, and charted paths for their mitigation and future work.
\end{abstract}

\section{Introduction}
Dagstuhl seminar 16251 on ``Information-centric Networking and Security'' was a three-day
workshop held on June 19-21, 2016 in Dagstuhl, Germany. The goal was to bring together researchers with
different areas of expertise relevant to ICN to discuss security and privacy issues
particular to ICN-based architectures. These problems have become increasingly
important as ICN technology gradually matures and gets close to real-world deployment.

\paragraph{Brief History of Dagstuhl Seminars on ICN}
This seminar is the fourth retreat of the ICN community at Schloss Dagstuhl, who
usually meet every two years at this place. In 2010, the term ICN was
coined to assemble different informa\-tion-centric networking approaches,
such as 4WARD, SAIL, PSIRP in Europe and content-centric networking (CCN) in the US.
In 2012, reality checks have been discussed. In 2014, special focus was on scalability
and deployment issues. This seminar discussed primarily security aspects.

\paragraph{Current Research Challenges}
Threat models in ICN are distinct from traditional IP-based networks ~\cite{wsv-bdpts-13,gtuz-ddndn-13}. Differentiating factors between the two include
new application design patterns, trust models and management, as well as a strong
emphasis on object-based, instead of channel-based, security. As ICN develops, it is both
timely and important to explore ICN security and privacy issues in order to devise and
assess possible mitigation techniques. This was the general purpose of the Dagstuhl
seminar. To that end, the seminar focused on the following issues:
%
\begin{compactitem}
\item What are the relevant threat models that ICN must be concerned? How are
they different from those in the IP-based networks?
\item To what extent is trust management a solved problem in ICN? Have we adequately
identified the core elements of a trust model, e.g., with NDN trust schemas?
\item How practical and realistic is object-based security when framed in the
context of accepted privacy measures used in the IP-based networks?
\item Are there any new types of cryptographic schemes or primitives that ICN architectures
could use to enable (a) more efficient or secure packet
processing or (b) a more security architecture?
\end{compactitem}
%
The seminar answered (entirely or partially) some of these questions and fueled discussions
for others. The major takeaways that resulted from these discussions are as follows.

\noindent
{\bf Management of namespace and identity}. The ICN community still does not have a clear answer on how to handle
namespace and identity management. While trust management in ICN can be distributed and function without a global PKI, it seems difficult to break away from this model for namespace management and arbitration. This has strong implications on how names are
propagated in the routing fabric. Can any producer application advertise any name,
anywhere in the network? If not, how can name prefix advertisements be constrained or limited? \\

\noindent
{\bf Forward secrecy}. Given that ICN focuses on object security, the need for transport
protocols that provide forward secrecy should be implemented in higher layers. Attendees
found that while most ICN-based architectures do not preclude forward secrecy, forward secrecy should
not be a requirement in the network layer.\\

\noindent
{\bf Locator and identifier split}. There is still a high uncertainty about whether ICN should split content
locator and identifier. Names in architectures such as NDN and CCN serve as both
locator and identifier of the data, though there are extensions that permit explicit locators
(e.g., through the use of NDN LINK objects). This distinction is necessary under the common understanding that
routing could be more efficient with topological names. Finding
data through non-topological names should not be implemented in the data plane as part of the global
routing space. However, if we revert to a distinction between topological locators and
identifiers, then the unique features of ICN become much more limited.
%One facet that is certainly unique to ICN is how software is written. Specifically, we have the opportunity to move beyond the mental model of a fixed address space and re-design existing network stacks and APIs. @@@ Edith: I don't quite understand the above sentences (commented). How does routing relate to software? @@@ \\

\noindent
{\bf Privacy tradeoffs}. Privacy seems difficult to achieve without major architectural changes to
ICN-based systems. In particular, since data names reveal large amount of information to
the passive eavesdropper, privacy demands that names and payloads have no correlation.
However, achieving this seems unfeasible without the presence of an upper-layer service
akin to the one that would resolve non-topological identifiers to topological names.\\

\noindent
{\bf How much crypto}. There are no compelling reasons to apply esoteric (and often untested)
cryptographic techniques in ICN, at least at the network layer. Computationally bounded
and traditional cryptographic primitives, such as elliptic curve digital signatures, hash functions, etc.,
could be the extent of per-packet cryptographic processing done by the routers.
%Anything more would become fodder for Denial-of-Service attacks that could render the entire infrastructure ineffective.
However, architecture designs should not restrict themselves
to specific algorithms. There should be flexibility in accommodating
multiple (and evolving) cryptographic primitives. This could be useful if, for example,
post-quantum digital signature schemes become necessary for the longevity of content authenticators.

The remainder of this workshop report is structured as follows.
Section~\ref{sec:plenary-talks} summarizes plenary talks.
Section~\ref{sec:group-work} presents discussions during group work in detail.
Finally, Section~\ref{sec:open-problems} and Section~\ref{sec:outlook} discuss open problems and future work, respectively.

\section{Summary of Plenary Talks} \label{sec:plenary-talks}
The main seminar material was driven by plenary talks on a variety of topics
ranging, such as trust management, namespace management, privacy, and anonymity.
These talks began with a discussion of threat models and their importance in ICN.
In particular, recall that ICNs attempt to diverge from IP with respect
to the central abstraction of hosts and point-to-point communication between
them. The ICN emphasis on named data and object security instead channel security
is one clear differentiating factor contributing to this parity. To quantify
the degree by which secret is improved (or worsened), threat models are needed.
In general, they must capture a particular design challenge in ICN, such as
infrastructure protection, user-friendly key distribution and trust management
and enforcement, and content protection and access control. And given the wide gap
between IP and ICN, there is a great need for common threat models to use
in the design phrase.

One particular threat revolves around consumer anonymity, which was also
the topic of discussion at the seminar. The subject(s) and contents
of a packet are not the only facets that should be considered with respect to privacy.
Origin and destination details (e.g., geographical location or position within a network
topology), as well as identity information (e.g., consumer identifying information),
can sometimes harm network users. As shown by \cite{compagno2015violating}, ICN caching and interest-collapse mechanisms make
ICN itself inherently vulnerable to the possibility for an adversary to locate
consumers. Moreover, an approach similar to the one to violate consumer
(location) privacy, might be used also to detect eavesdropper. Therefore, the
threat model must consider this vulnerability and adversaries capable of exploiting
it successfully.

Another design challenge unique to ICN is how to develop scalable object-based
access control mechanisms. A variety of encryption techniques have been used
in the past to protect access to confidential network data \cite{tourani2016security}.
Many design approaches, particularly in CCN and NDN, exist well above the network
layer. In contrast, publish-subscribe ICNs such as ENCODERS \cite{raykova2015decentralized} integrate
access control into the network. It uses multi-authority attribute-based encryption
to allow content access to be scoped to selected nodes in the system. Since the
system is completely decentralized, peers serve as brokers that match content from
publishers with interests expressed by subscribers. In order to perform such a match,
an intermediate node must be authorized to see both the relevant content tags and
subscriber interests. One talk at the seminar focused on the observation that
access control policies applied to the metadata (content tags and subscriber interests)
effectively create reachability constraints that are independent from the one defined by
the routing protocols. Consequently, this security-routing interaction must be
treated carefully during policy definition.

Another major category of plenary talks focused on the future of ICNs. There are many infrastructure
security and privacy problems in IP-based networks that we could try to remedy. 
Can ICNs use the NSEC3 strategy of \emph{authenticated denial} \cite{blacka2008dns} to limit incoming requests
for non-existent content as a way of deterring sophisticated distributed DoS (DDoS) attacks?
Can we levereage recent advances in deep packet inspection over encryption data \cite{sherry2015blindbox}
to let forwarders blindly route packets without seeing cleartext or otherwise sensitive information,
e.g., application names? Can we leverage information-theoretic PIR techniques to allow for 
truly private content queries (similar to what is done in \cite{pir-icn}). Can we prevent correlation
of static content across multiple consumers by adopting randomizable encryption schemes \cite{blazy2011signatures}? 
And if so, how can we do so while maintaining the integrity of content. There are many 
features that could improve beyond what is done in IP-based networks and there are cryptographic
algorithms, schemes, and protocols that could allow us to realize these features. However,
it is unclear whether or not these more esoteric cryptographic schemes should be applied in
the network layer of a future Internet. 


%%% Q: how to rope in PANINI, Project Origin, etc here?

\section{Parallel Group Work} \label{sec:group-work}
\subsection{ICN and IoT}

The Internet of Things (IoT) is connecting billions of smart devices (e.g., sensors) and is growing very fast. More than one million networked ``things'' will be expected per square kilometer in 2030. This work group explored how much data density IoT will expect and and how the communication will look like with the ``things''. For example, users can communicate directly to the sensors, or indirectly through the cloud or gateways. Considering the characteristics and trends of IoT, the group discussed the potential and benefits of implementing ICN for the IoT. For instance, the ICN routers connecting to sensors can cache the sensor data to improve the performance of data dissemination. In addition, the users can obtain sensor data directly from the sensors or from the ICN routers, without going through the cloud. Although ICN for IoT may provide in-network caching and flexibility in data dissemination, it raises several security concerns:

\begin{itemize}
\item How can the sensors be securely configured at the time of initialization?
\item How can software updates be performed securely?
\item How can we handle mobility in ICN for IoT? For example, each sensor may need a unique publisher identity, which may or may not change with its location. How does mobility affect naming of data and scalability of routing?
\end{itemize}

The group also discussed the concerns of caching data at the sensors. Firstly, sensors are resource limited devices, which may not have sufficient memory for caching the data. However, memory resources in the sensors may increase and the price may go down in the future. Secondly, it is advantageous to retrieve data directly from the sensors in certain use cases. For example, it is more efficient to control home lighting without going through the cloud. Thirdly, when using cryptography on  the sensors, the encryption time could be long and cause additional delay in data retrieval. Lastly, the sensors have to be always on to listen to the interest packets in the ICN, which may consume a lot of energy. Scheduling or adaptive duty cycles might be considered to mitigate this problem.

Based on these observations, the group summarized the discussion and layout the future plan as follows. Firstly, the group plans to  select and explore a few use cases in the IoT, which allow better understanding on the communication patterns and the security requirements in ICN for IoT. Secondly, it aims at investigating systemically on the following research questions: How does IoT benefit from ICN? What is the best way to securely configure and bootstrap the sensors? Finally, what is the cost of providing security for IoT data with ICN?

\subsection{Namespace and Identity Management}
This group discussed the relation between trust, identity, and namespace management.
All three aspects are closely tied in ICN.
Furthermore, naming, which in the current Internet is primarily located on the application layer, now directly affects the network layer as well.

In an ideal ICN architecture, applications should be able to express their trust preferences (using policies) and let some ``middleware'' enforce them throughout the network. This raises two important questions: (1) what is the minimum set of policies that can be factored out of all trust models, and (2) what is the middleware that does this enforcement? The trust schemas pioneered by the NDN architecture \cite{schemas} are exemplary of the common rules that can be used to express most trust models. Among other things, they specify what keys are allowed to sign what data. Since both keys and data are named resources in NDN and other ICN architectures, this means that a schema allows for arbitrary hierarchical trust models. It remains to be seen if other non-hierarchical trust models will be applicable to ICN.

To address the second question, the group had to agree upon the responsibility of the network with respect to enforcing policies, details are discussed in \cite{trust}.
First and foremost, network layer ``trust enforcement'' should not prohibit or prevent other app\-lication-layer trust models. This means that the network functionality must be simpler than that which is supported by the middleware. Currently, this is comprised of (at most) digital signature and content object hash verification. Functionalities such as certificate chain resolution or key retrieval should not be implemented in the core network. This means that in the backbone ``network,'' routers are only responsible for single signature or hash verification. All other network nodes (e.g., consumers and producers) run the middleware responsible for handling the remaining trust enforcement steps. It is worth noting that end systems are supersets of routers in terms of functionality, and thus may route data as well.

After addressing network trust, the group turned to identity and discussed the following major questions:
%
\begin{itemize}
\item How are names registered and managed in ICN?
\item How can names possibly be location agnostic (without aids such as the NDN LINK)? Is there always a discovery or locator service?
\end{itemize}
%
Namespace ownership is intrinsically tied to identity. Thus, namespace advertisements under different namespaces or in different networks must be authenticated with respect to the claimed owner's identity. In this context, an identity is a public and private key pair. The community struggled with issues about namespace scale and the practicality of a global namespace. Questions such as, ``how do NATs work in a global namespace?,'' drove the discussion. No consensus or common understanding about how namespaces and identities should be managed was reached.


\subsection{Routing on Encrypted Names}
This group started with a discussion about routing on encrypted names but ended up being an exploration of name privacy and the necessary conditions for it to be possible in different ICN architectures. In this context, we defined name privacy to be the property where a so-called ``network name,'' i.e., the name encoded in a packet, has no correlation or connection with the corresponding content object. Specifically, name privacy means that a network name reveals nothing about the data inside the content object. Ideally, names should reveal no more than what is currently revealed by an IP address and port. After settling on this definition, we laid out our assumptions to use when discussing name privacy, including:

\begin{itemize}
\item There is no name discovery process or search engine.
\item Content may be requested by an identifier (ID) such as its cryptographic hash digest. Moreover, revealing the content ID does not compromise privacy.
\item Consumers know the public key of the producer with which they want to communicate.
\item Network names have an implicit routable prefix and application-specific suffix. By default, consumers do not know the locator and identifier split in a name.
\item Requests may specify the ID of (1) a signature verification key or (2) the expected content.
\end{itemize}

To begin, there are fundamentally two ways to request content: (1) with and (2) without a content ID. In case (1), a request name needs to only contain a routable prefix that will move the request to some cache or producer which can return the corresponding content. These locators can be completely separate from the desired content and, therefore, this approach can satisfy our name privacy goal. However, without implicit knowledge about the locator for some desired content, an upper-layer service is necessary to obtain said information.

In case (2), the application-specific suffix of a name must not reveal anything about the data. To achieve this, it must be encrypted. Name encryption introduces a number of other questions, such as how to obtain the routable prefix, what key to use for encryption, and how to ``protect'' the result. Assume that the routable prefix is known and that the producer public key is used for name suffix encryption. If the resultant content payload is not encrypted then one may be able to infer the name from its contents. Therefore, the content response itself must also be encrypted. This requires requests to carry a consumer-generated key that is protected in a CCA-secure envelope. Otherwise, eavesdroppers could replay requests with the same encrypted name but their own key to obtain a decrypted response.

In all cases, we concluded that to achieve name privacy then one needs some upper-layer service. Whether its role is to provide the routable prefix for a name, encrypt the response, or to separate a content ID and locator via some other means is an orthogonal discussion. Also, name privacy seems to, in most cases, invalidate the utility of shared caches, which puts it at odds with the primary feature of many ICN-based architectures. Thus, our conjecture is that name privacy is not possible natively in the network.


\subsection{Locators and Identifiers}
This working group discussed locators and fetching data with non-topological names (or even topological names that are cached off-path). Routing should, it seems, only concern itself with topological names or addresses.  Finding data (objects) with non-topological names should not be done in the data plane.  It should be done via a service.

In CCN, the service could resolve a named root manifest to then resolve locator names by hash.
In NDN, it resolves the link routing hints to allow off-path interest forwarding. In TagNet,
there is a distinction between Locator names and Descriptor names. Locator names have a
strong binding between their name and a point of attachment. Descriptor (hash) names, on the
other hand, are free-form and could be present anywhere.  One resolves a tag query (of either
type) to a topological locator and then does data transfer on that locator.

This lead to a discussion on locator and identifier split.  Should CCN embrace this, or continue on with its mixed use of the name? For example, if there is a clear locator field and then a clear identifier tuple (name, [keyed restriction], [hash restriction]), one would get full matching expressivity with the functionality of nameless object locators.  A similar approach could be done in NDN, though with a different tuple.  There was no consensus on this idea, though it is worth exploring.

There was also some discussion on the benefit of ICN if one still needs to do an external name to address lookup.  Why bother if one still needs a DNS-like function?  One partial answer is that in the non-global routing space (i.e., data center, maybe IoT, some internal applications), one could inject all names into the internal routing protocol and realize the full benefit of application-specific names.  Another argument is that it improves how one writes software to not have to deal with IP addresses and host-based networking. One could also see benefits from a re-designed network stack beyond sockets.

\subsection{Security, Not Privacy}
The ``No Privacy'' security working group sought to answer the following question: is an ICN security architecture easier to devise if the designs fundamentally make privacy hard to achieve? In particular, the group discussed:

\begin{itemize}
\item What ICN entities (content consumers, hosts, routers, content creators) need identities?
\item What entities can simply operate with a public and private key pair but no formal name?
\item Does splitting routing out as an application help?
\item Do interests need to be authenticated at each router?
\end{itemize}

The group achieved a simple security model. Members of the group hope to write up the result as a short paper.

\section{Open Problems}
In this section we highlight several open problems that loomed large at the
seminar. They should be considered in future research endeavors.

\subsection{Privacy}
What is the future of privacy for ICN? To what, or whom, is ICN privacy related?
Existing architectures leak a significant amount of information by default,
including: who requests information, whose information is requested, when content
is requested, and other miscellaneous properties, e.g., data contents, name, size,
etc. As of yet, we have not adequately addressed these privacy problems.
Beyond ICN packets, ICN programmers will generate linked data, e.g., FLIC [1]. We
should provide techniques, services, and recipes for programmers that make transport
privacy a trivial. Before doing so, however, we must first define what is transport
privacy in the context of ICN.

\subsection{Forward Object Secrecy?}
We revisit the question of securing the data, not the pipe. This has been the
running mantra for security in ICN. Is forward secrecy possible in NDN and CCN
group access control by encryption? Is the ``take what you want'' model of group
access control, dependent on long-lived keys, realistic for future networks? Is
it desirable? Also, is there any role for perimeter security or is encryption
enough? In this talk, we pose these questions and others to the group to stimulate
a wider discussion.

Forward secrecy is the property that exposure of a principal's long-term secret keys
does not compromise the secrecy of their previously generated ephemeral (session)
keys. This is a useful property to have in the presence of eavesdropping attackers
intercepting and logging traffic. It minimizes data and key compromise windows and
therefore reduces the overall attack surface. However, it requires protocols and
techniques for deriving ephemeral keys and then updating them regularly. The single
request-response model of many ICN-based architectures does not lend itself to the
establishment of forward secrecy without building a higher-layer protocol, such as
CCNxKE [1], or involving more exotic cryptographic schemes. Consequently, the
majority of work on ICN object security has ignored this property, which puts ICN
at odds with best practice techniques used in IP-based protocols. In this talk, we
seek to raise awareness of this issue and seek answers to the following important
questions. First, under what conditions does transport security require forward
secrecy? Second, can object encryption subsume transport security? And lastly, is
forward secrecy in ICN needed?

\subsection{Names and Routing}
ICN names are user-generated content in FIBs. In effect, FIBs serve as a (globally)
replicated name set wherein any name owner can write into the set. The complexity of
this state is influenced by the fact that prefix owners can always de-aggregate and
create arbitrary names, even if prefixes are restrictively assigned. However, this
raises questions of resource exhaustion attacks on FIBs and general complexity
attacks (e.g., hash collisions). Newer attacks try to leak information from the
FIB contents to target the forwarding plane. This talk outlines the severity of
these problems in hopes of discussing potential solutions.

\subsection{Coping Network Services}
The Internet has a history of adapting the existing law system to new business paradigms.
One such paradigm is in-network processing, which, in recent years, has expanded to
aid and impact routing, forwarding, packet replication, packet splitting and merging,
quality of control, caching, and others. The relationship between these services and
existing laws has been a continual tussle. When and how does caching affect copyright laws?
When do other services violate the Secrecy of Correspondence (SoC) statute?
Moreover, Deep Packet Inspection on SSL/TLS connections, while technically feasible, may violate
the SoC statute and various other privacy rules. Thus, there has been a recent push for
all-or-nothing secrecy, which unfortunately stifles network business opportunities.
In this talk, we advocate for controllable privacy that allows secrecy preferences to be
expressed in packet headers. We claim that ICN packet headers should be constructed to allow
privacy and secrecy preferences to be expressed by their senders. This is one area where ICN
can innovate to allow in-network processing to continue without violating existing laws.

\section{Outlook and Open Problems} \label{sec:outlook}
After nearly a decade of research, there still exists an abundance of security
and privacy problems that the ICN community has yet to adequately address.
This leaves much room for future work on a variety of topics, which are elaborated
upon below.

\subsection{Object-Based Security with Forward Secrecy}
{\bf Summary}. We need to resolve the channel- vs object-based security debate.
Should we continue to focus on securing the data instead of securing the pipe?
The former promotes a ``take what you want'' model of group access control which
is dependent on long-lived keys.
One primary factor why channel-based security is preferred is because it allows for forward
secrecy. Forward secrecy is the property that exposure of a principal's long-term secret keys
does not compromise the secrecy of their previously generated ephemeral (session)
keys. This is a useful property to have in the presence of eavesdropping attackers
intercepting and logging traffic. It minimizes data and key compromise windows and
therefore reduces the overall attack surface. However, it requires protocols and
techniques for deriving ephemeral keys and then updating them regularly.
Without forward secrecy, the packet confidentiality is reduced to the efficacy of key management.
If private keys are leaked, then intercepted packets can be decrypted.
Transport layer security adds another obstacle for attackers. If we can design an object-based security scheme that
also provides forward secrecy, say built on the forward-secure public-key
encryption scheme of Canetti et al. \cite{canetti2003forward}, then this argument is simplified.

Given the IP-based push for TLS everywhere that leads to applications built on top of (D)TLS
rather than plain TCP (or UDP), this type of communication is a significant departure from
what is accepted as best practice today. Can ICN-based architectures make a compelling
enough case to motivate applications to revert to this less-secure form of
data delivery?

{\bf Outlook}. Given that ICN focuses on object security, the need for transport
protocols that provide forward secrecy should be implemented in higher layers. Attendees
found that while most ICN-based architectures do not preclude forward secrecy, forward secrecy should
not be a requirement in the network layer.\\

\subsection{Namespaces, Identities, and Routing}
{\bf Summary}. In ICNs, namespace ownership, identities, and
the routing fabric are all intimately coupled. Currently, we do not know of a way
to break away from a centralized model for namespace management and arbitration.
Routing updates therefore have a dependency on this management oracle. Otherwise,
any producer application would be able to advertise any namespace it wants. Schemes
such as \cite{dibenedetto2015mitigating} can be used to remove malicious producers
from advertising under incorrect namespaces, but it \emph{does not} resolve conflicts
over namespace ownership.

{\bf Outlook}. The ICN community still does not have a clear answer on how to handle
namespace and identity management. While trust management in ICN can be distributed and
function without a global PKI, it seems difficult to break away from this model for
namespace management and arbitration. This has strong implications on how names are
propagated in the routing fabric. Can any producer application advertise any name,
anywhere in the network? If not, how can name prefix advertisements be constrained or limited?

\subsection{Network Management}
{\bf Summary}. ICN names are user-generated content in FIBs. In effect, FIBs serve as a (globally)
replicated name set wherein any name owner can write into the set. The complexity of
this state is influenced by the fact that prefix owners can always de-aggregate and
create arbitrary names, even if prefixes are restrictively assigned. However, this
raises questions of resource exhaustion attacks on FIBs and general complexity
attacks (e.g., hash collisions). Newer attacks try to leak information from the
FIB contents to target the forwarding plane. Can ICNs be managed to avoid these
types of scalability or security problems, or do they necessitate an ecosystem in
which any producer can inject any type of information into the network state.

{\bf Outlook}. This class of problem is tied to how namespaces, identities, and routing
are to be handled in ICN-based architectures. Until we have a shared understanding
of how namespace ownership, advertisement, and propagation will be controlled, we
cannot expect to manage the network state to prevent state exhaustion or mitigate
FIB scalability problems.

\subsection{Evolving Network Services}
{\bf Summary}. The Internet has a history of adapting the existing law system to new business paradigms.
One such paradigm is in-network processing, which, in recent years, has expanded to
aid and impact routing, forwarding, packet replication, packet splitting and merging,
quality of control, caching, and others. The relationship between these services and
existing laws has been a continual tussle. When and how does caching affect copyright laws?
When do other services violate the Secrecy of Correspondence (SoC) statute?
Moreover, Deep Packet Inspection on SSL/TLS connections, while technically feasible, may violate
the SoC statute and various other privacy rules. Thus, there has been a recent push for
all-or-nothing secrecy, which unfortunately stifles network business opportunities.

{\bf Outlook}. Privacy should be controllable in that it allows secrecy preferences to be
expressed by senders in packet headers. This is one area where ICN can innovate to
allow in-network processing to continue without violating existing laws. This type
of flexibility is not possible in IP-based protocols such as TCP and TLS, both
of which have fixed packet frames and do not easily permit any sort of privacy expression.
One notable exception to this claim is the presence of TCP option fields. These may
be (mis)used to allow for privacy preferences to be conveyed. For example, tcpcrypt \cite{ietf-tcpinc-tcpcrypt-03}
uses these option fields to encode key exchange parameters used to establish
a secure channel between both ends of the TCP connection. So, while it may be feasible
to express privacy preferences in some packet headers, existing protocols were certainly
not designed with this in mind.

\subsection{Privacy}
{\bf Summary}. Privacy of consumers, producers, and content all remain significant
challenges in most ICN-based architectures. Moreover, the trend in the IETF and other
standard bodies is to put privacy as a \emph{primary} goal going forward. Consequently,
to meet future privacy expectations, many architectures need to make certain techniques
such as onion-based routing \cite{uzun2011anonymous}, name encryption \cite{privacy}, or secure
sessions \cite{wood-icnrg-ccnxkeyexchange-00} a de facto part of the architecture.
Otherwise, it is difficult to foresee the incentives to use these architectures in the
real world.

{\bf Outlook}. Privacy seems difficult to achieve without major architectural changes to
ICN-based systems. Existing architectures leak a significant amount of information by default,
including: who requests information, whose information is requested, when content
is requested, and other miscellaneous properties, e.g., data contents, name, size,
etc. As of yet, we have not adequately addressed these privacy problems.
In particular, since data names reveal large amount of information to
the passive eavesdropper, privacy demands that names and payloads have no correlation.
However, achieving this seems unfeasible without the presence of an upper-layer service
akin to the one that would resolve non-topological identifiers to topological names.\\

\subsection{Locator and Identifier Split}
{\bf Summary}. By focusing on named data instead of hosts, many ICN-based architectures
blur the line between content locators and identifiers. This has implications on how
routing and discovery occurs in the network. Existing architectures differ in packet format
and protocol semantics in how these two mechanisms are performed. More research is
needed before we, as a community, declare one approach superior to another.

{\bf Outlook}. There is still a high uncertainty about whether ICN should split the content
locator and identifier. Names in architectures such as NDN and CCN serve as both
locator and identifier of the data, though there are extensions that permit explicit locators
(e.g., through the use of NDN LINK objects). This distinction is necessary under the common understanding that
routing could be more efficient with topological names. Finding
data through non-topological names should not be implemented in the data plane as part of the global
routing space. However, if we revert to a distinction between topological locators and
identifiers, then the unique features of ICN become much more limited.
One facet that is certainly unique to ICN is how software is written. Specifically, we
have the opportunity to move beyond the mental model of a fixed address space and re-design
existing network stacks and APIs.

\subsection{Common Crypto}
{\bf Summary}. ICN-based architectures have the unique privilege of being able to start from
a clean slate without an inheriting any legacy cruft. This often leads to a strong
desire to explore the use of young cryptographic primitives and protocols, such as
those built on pairings \cite{wood2014flexible}. The mistake designers often make is
that certain architectural features or system characteristics become dependent on
these cryptographic schemes. Thus, in the off-chance that they should be found insecure,
then the architecture or system is no longer valid. The use of such cryptographic
techniques is dubious at best. The security and cryptography communities need
more time to assess emerging primitives and protocols before they are adopted in
any major way. One avenue is to funnel designs through the CFRG \cite{cfrg}, which
is often responsible for bridging the gap between academia and industry. Recent
schemes under consideration by this research group include password-authenticated key
exchange protocols and post-quantum-secure hash-based signature schemes. The latter
of which is particularly relevant with respect to content authenticators.

{\bf Outlook}. There are no compelling reasons to apply esoteric (and often immature)
cryptographic techniques in ICN, at least at the network layer. Computationally bounded
and traditional cryptographic primitives, such as elliptic curve digital signatures, hash functions, etc.,
could be the extent of per-packet cryptographic processing done by the routers.
Anything more would become fodder for Denial-of-Service attacks that could render
the entire infrastructure ineffective. However, architecture designs should not
restrict themselves to specific algorithms. There should be flexibility in accommodating
multiple (and evolving) cryptographic primitives. This could be useful if, for example,
post-quantum digital signature schemes become necessary for the longevity of content authenticators.


\section{Conclusion}
This paper described in detail the events and outcomes of the Dagstuhl 16251
seminar on ICN security and privacy. Despite significant research over the past
half decade, there are still many open problems with solutions that are difficult
to be completely realized with the existing architectures. Are we too invested in the current architectures to make significant design changes to solve these problems? Is there something to be gained by sacrificing properties such as privacy in favor of features such as object security? If so, is this the right tradeoff to make today? Only future research and development will tell.

\subsection*{Acknowledgements \& Participants}
We thank Schloss Dagstuhl for providing a stimulating setting for this seminar. Much
progress was made over the course of the seminar and since its completion. This is mainly
because of the ease of face-to-face collaboration and interaction at Dagstuhl.

\begin{comment}
Furthermore, we thank all participants for fruitful \emph{and} open discussions
on very urgent challenges. The following people attend Dagstuhl Seminar 16251:
    Bengt Ahlgren (Swedish Institute of Computer Science, SE),
    Tohru Asami (University of Tokyo, JP),
    Roland Bless (Karlsruher Institut f\"ur Technologie, DE),
    Randy Bush (Internet Initiative Japan Inc., Tokyo, JP),
    Kenneth L. Calvert (University of Kentucky, Lexington, US),
    Antonio Carzaniga (University of Lugano, CH),
    Mauro Conti (University of Padova, IT),
    Lars Eggert (NetApp Deutschland GmbH, Kirchheim, DE),
    Darleen Fisher (NSF, Arlington, US),
    Ashish Gehani (SRI, Menlo Park, US),
    Jussi Kangasharju (University of Helsinki, FI),
    Ghassan Kar\-ame (NEC Laboratories Europe, Heidelberg, DE),
    Dirk Kut\-scher (NEC Laboratories Europe, Heidelberg, DE),
    John Mattsson (Ericsson Research, Stockholm, SE),
    Marc Mosko (Xerox PARC, Palo Alto, US),
    Edith Ngai (Uppsala University, SE),
    B\"orje Ohlman (Ericsson Research, Stockholm, SE),
    J\"org Ott (TU M\"unchen, DE),
    Craig Partridge (BBN Technologies, Cambridge, US),
    Fabio Pianese (Bell Labs, Nozay, FR),
    Sanjiva Prasad (Indian Inst. of Technology, Dehli, IN),
    Thomas C. Schmidt (HAW Hamburg, Hamburg, DE),
    Sebastian Sch�nberg (Intel, Santa Clara, US),
    Christoph Schuba (Ericsson, San Jose, US),
    Glenn Scott (Xerox PARC, Palo Alto, US),
    Jan Seedorf (NEC Laboratories Europe, Heidelberg \& Hochschule f\"ur Technik Stuttgart, DE),
    Tim Strayer (BBN Technologies, Cambridge, US),
    Christian Tschudin (Universit\"at Basel, CH),
    Gene Tsudik (University of California, Irvine, US),
    Ersin Uzun (Xerox PARC, Palo Alto, US),
    Matthias W\"ahlisch (FU Berlin, DE),
    Cedric Westphal (Huawei Technologies, Santa Clara, US),
    Christopher A. Wood (University of California, Irvine, US).
\end{comment}

%\small
\balance
\bibliographystyle{IEEEtran}
\bibliography{IEEEabrv,references}

\end{document}

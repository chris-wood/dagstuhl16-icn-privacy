\section{Open Problems}
\label{sec:open-problems}

In this section we highlight several open problems that have attracted much attention at the
seminar. They should be considered in future research endeavors.

\subsection{Privacy}
What is the future of privacy for ICN? To what, or whom, is ICN privacy related?
Existing architectures leak a significant amount of information by default,
including: who requests information, whose information is requested, when content
is requested, and other miscellaneous properties, e.g., data contents, name, size,
etc. As of yet, we have not adequately addressed these privacy problems.
Beyond ICN packets, ICN programmers will generate linked data, e.g., FLIC [1]. There is a need to
 provide techniques, services, and recipes for programmers to make transport
privacy a trivial. Before doing so, however, it is important to first define what is transport
privacy in the context of ICN.

\subsection{Forward Object Secrecy} %?
We revisit the question of securing the data, not the pipe. This has been the
running mantra for security in ICN. Is forward secrecy possible in NDN and CCN
group access control by encryption? Is the ``take what you want'' model of group
access control, dependent on long-lived keys, realistic and desirable for future networks?
 Also, is there any role for perimeter security or is encryption
enough? In this talk, we pose these questions and others to the group to stimulate
a wider discussion.

Forward secrecy is the property that exposure of a principal's long-term secret keys
does not compromise the secrecy of their previously generated ephemeral (session)
keys. This is a useful property to have in the presence of eavesdropping attackers
intercepting and logging traffic. It minimizes data and key compromise windows and
therefore reduces the overall attack surface. However, it requires protocols and
techniques for deriving ephemeral keys and then updating them regularly. The single
request-response model of many ICN-based architectures does not lend itself to the
establishment of forward secrecy without building a higher-layer protocol, such as
CCNxKE [1], or involving more exotic cryptographic schemes. Consequently, the
majority of work on ICN object security has ignored this property, which puts ICN
at odds with best practice techniques used in IP-based protocols. In this talk, we
seek to raise awareness of this issue and seek answers to the following important
questions. First, under what conditions does transport security require forward
secrecy? Second, can object encryption subsume transport security? And lastly, is
forward secrecy in ICN needed?

\subsection{Names and Routing}
ICN names are user-generated content in FIBs. In effect, FIBs serve as a (globally)
replicated name set wherein any name owner can write into the set. The complexity of
this state is influenced by the fact that prefix owners can always de-aggregate and
create arbitrary names, even if prefixes are restrictively assigned. However, this
raises questions of resource exhaustion attacks on FIBs and general complexity
attacks (e.g., hash collisions). Newer attacks try to leak information from the
FIB contents to target the forwarding plane. This talk outlines the severity of
these problems in hopes of discussing potential solutions.

\subsection{Coping Network Services}
The Internet has a history of adapting the existing law system to new business paradigms.
One such paradigm is in-network processing, which, in recent years, has expanded to
aid and impact routing, forwarding, packet replication, packet splitting and merging,
quality of control, caching, and others. The relationship between these services and
existing laws has been a continual tussle. When and how does caching affect copyright laws?
When do other services violate the Secrecy of Correspondence (SoC) statute?
Moreover, Deep Packet Inspection on SSL/TLS connections, while technically feasible, may violate
the SoC statute and various other privacy rules. Thus, there has been a recent push for
all-or-nothing secrecy, which unfortunately stifles network business opportunities.
In this talk, we advocate for controllable privacy that allows secrecy preferences to be
expressed in packet headers. We claim that ICN packet headers should be constructed to allow
privacy and secrecy preferences to be expressed by their senders. This is one area where ICN
can innovate to allow in-network processing to continue without violating existing laws.

\section{Outlook and Open Problems} \label{sec:outlook}
After nearly a decade of research, there still exists an abundance of security
and privacy problems that the ICN community has yet to adequately address.
This leaves much room for future work on a variety of topics, which are elaborated
upon below. 

\subsection{Object-Based Security with Forward Secrecy}
{\bf Summary}. We need to resolve the channel- versus object-based security debate.
Should we continue to focus on securing the data instead of securing the pipe?
The former promotes a ``take what you want'' model of group access control which
is dependent on long-lived keys.
One factor why channel-based security is preferred is because it allows for forward
secrecy. Forward secrecy is the property that exposure of a principal's long-term secret keys
does not compromise the secrecy of their previously generated ephemeral (session)
keys. This is a useful property to have in the presence of eavesdropping attackers
intercepting and logging traffic. It minimizes data and key compromise windows and
therefore reduces the overall attack surface. However, it requires protocols and
techniques for deriving ephemeral keys and then updating them regularly.
Without forward secrecy, packet confidentiality is reduced to the efficacy of key management.
That is, if private keys are leaked, then intercepted packets can be decrypted.

Given the IP-based push for TLS everywhere that leads to applications built on top of (D)TLS
rather than plain TCP (or UDP), object-based security is a significant departure from
what is accepted as best-practice today. Can ICN-based architectures make a compelling
enough case to motivate applications to revert to this less-secure form of
data delivery?

{\bf Outlook}. Given that ICN focuses on object security, the need for transport
protocols that provide forward secrecy should be implemented in higher layers. Attendees
found that while most ICN-based architectures do not preclude forward secrecy, forward secrecy should
not be a requirement in the network layer. However, a worthy research question is
whether or not we can design an object-based security scheme that
also provides forward secrecy. Reviewing existing schemes such as the forward-secure public-key
encryption scheme of Canetti et al. \cite{canetti2003forward} might prove useful here.

\subsection{Namespaces, Identities, and Routing}
{\bf Summary}. As previously discussed, in ICNs, namespace ownership, identities, and
the routing fabric are all intimately coupled. Currently, we do not know of a way
to break away from a centralized model for namespace management and arbitration.
Routing updates therefore have a dependency on this management oracle. Without it,
any producer application would be able to advertise any namespace it wants. Schemes
such as \cite{dibenedetto2015mitigating} can be used to remove malicious producers
from advertising under incorrect namespaces, but they \emph{do not} resolve conflicts
over namespace ownership.

{\bf Outlook}. The ICN community still does not have a clear answer on how to handle
namespace and identity management. While trust management in ICN can be distributed and
function without a global PKI, it seems difficult to break away from this model for
namespace management and arbitration. This has strong implications on how names are
propagated in the routing fabric. Can any producer application advertise any name,
anywhere in the network? If not, how can name prefix advertisements be constrained or limited?

\subsection{Network Management}
{\bf Summary}. ICN names are user-generated content in FIBs. In effect, FIBs serve as a (globally)
replicated name set wherein any name owner can write into the set. The complexity of
this state is influenced by the fact that prefix owners can always de-aggregate and
create arbitrary names, even if prefixes are restrictively assigned. However, this
raises questions of resource exhaustion attacks on FIBs and general complexity
attacks (e.g., hash collisions). Newer attacks try to leak information from the
FIB contents to target the forwarding plane. Can ICNs be managed to avoid these
types of scalability or security problems, or do they necessitate an ecosystem in
which any producer can inject any type of information into the network state?

{\bf Outlook}. This class of problem is tied to how namespaces, identities, and routing
are to be handled in ICN-based architectures. Until we have a shared understanding
of how namespace ownership, advertisement, and propagation will be controlled, we
cannot expect to manage the network state to prevent state exhaustion or mitigate
FIB scalability problems.

\subsection{Privacy}
{\bf Summary}. Privacy of consumers, producers, and content all remain significant
challenges in most ICN-based architectures. As of yet, we have not adequately addressed
these privacy problems. In particular, since data names reveal large amount of information to
the passive eavesdropper, privacy demands that names and payloads have no correlation.
However, achieving this seems infeasible without the presence of an upper-layer service
akin to the one that would resolve non-topological identifiers to topological names.
Moreover, the trend in the IETF and other standard bodies is to put privacy as a \emph{primary} goal going forward. Consequently,
to meet future privacy expectations, many architectures may need to make certain techniques
such as onion-based routing \cite{uzun2011anonymous}, name encryption \cite{privacy}, or secure
sessions \cite{wood-icnrg-ccnxkeyexchange-01} a de facto part of the architecture.
Otherwise, it is difficult to foresee the incentives to use these architectures in the
real world.

{\bf Outlook}. Privacy seems difficult to achieve without major architectural changes to
ICN-based systems. As of now, it seems as though ICN-based architectures trade privacy
for efficiency, which contradicts the perspective of the IETF and beyond. More
research is needed to determine if both privacy and efficiency can be achieved in
future without such drastic tradeoffs.

\subsection{Locator and Identifier Split}
{\bf Summary}. There is still a high uncertainty about whether ICN should split the content
locator and identifier. Names in architectures such as NDN and CCN serve as both
locator and identifier of the data, though there are extensions that permit explicit locators
(e.g., through the use of NDN LINK objects). This distinction is necessary under the common understanding that
routing could be more efficient with topological names. Finding
data through non-topological names should not be implemented in the data plane as part of the global
routing space. However, if we revert to a distinction between topological locators and
identifiers, then the unique features of ICN become much more limited.
% One facet that is certainly unique to ICN is how software is written. Specifically, we
% have the opportunity to move beyond the mental model of a fixed address space and re-design
% existing network stacks and APIs.

{\bf Outlook}. By focusing on named data instead of hosts, many ICN-based architectures
blur the line between content locators and identifiers. This has implications on how
routing and discovery occurs in the network. Existing architectures differ in packet format
and protocol semantics in how these two mechanisms are performed. More research is
needed before we, as a community, declare one approach superior to another.

\subsection{Common Crypto}
{\bf Summary}. ICN-based architectures have the unique privilege of being able to start from
a clean slate without an inheriting any legacy cruft. This often leads to a strong
desire to explore the use of young cryptographic primitives and protocols, such as
those built on pairings \cite{wood2014flexible}. The mistake designers often make is
that certain architectural features or system characteristics become dependent on
these cryptographic schemes. Thus, in the off-chance that they should be found insecure,
then the architecture or system is no longer valid. The use of such cryptographic
techniques is dubious at best. The security and cryptography communities need
more time to assess emerging primitives and protocols before they are adopted in
any major way. One avenue is to funnel designs through the CFRG \cite{cfrg}, which
is often responsible for bridging the gap between academia and industry. Recent
schemes under consideration by this research group include password-authenticated key
exchange protocols and post-quantum-secure hash-based signature schemes. The latter
of which is particularly relevant with respect to content authenticators.

{\bf Outlook}. There are no compelling reasons to apply esoteric (and often immature)
cryptographic techniques in ICN, at least at the network layer. Computationally bounded
and traditional cryptographic primitives, such as elliptic curve digital signatures, hash functions, etc.,
could be the extent of per-packet cryptographic processing done by the routers.
Anything more would become fodder for Denial-of-Service attacks that could render
the entire infrastructure ineffective. However, architecture designs should not
restrict themselves to specific algorithms. There should be flexibility in accommodating
multiple (and evolving) cryptographic primitives. This could be useful if, for example,
post-quantum digital signature schemes become necessary for the longevity of content authenticators.

\subsection{Evolving Network Services}
{\bf Summary}. The Internet has a history of adapting the existing law system to new business paradigms.
One such paradigm is in-network processing, which, in recent years, has expanded to
aid and impact routing, forwarding, packet replication, packet splitting and merging,
quality of control, caching, and others. The relationship between these services and
existing laws has been a continual tussle. When and how does caching affect copyright laws?
When do other services violate the Secrecy of Correspondence (SoC) statute?
Moreover, Deep Packet Inspection on SSL/TLS connections, while technically feasible, may violate
the SoC statute and various other privacy rules. Thus, there has been a recent push for
all-or-nothing secrecy, which unfortunately stifles network business opportunities.

{\bf Outlook}. Privacy should be controllable in that it allows secrecy preferences to be
expressed by senders in packet headers. This is one area where ICN can innovate to
allow in-network processing to continue without violating existing laws. This type
of flexibility is not always possible in IP-based protocols such as TCP and TLS, both
of which have fixed packet frames and do not easily permit any sort of privacy expression.
One notable exception to this claim is the presence of TCP option fields. These may
be (mis)used to allow for privacy preferences to be conveyed. For example, tcpcrypt \cite{ietf-tcpinc-tcpcrypt-03}
uses these option fields to encode key exchange parameters used to establish
a secure channel between both ends of the TCP connection. So, while it may be feasible
to express privacy preferences in some packet headers, existing protocols were certainly
not designed with this in mind.

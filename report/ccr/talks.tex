\section{Summary of Plenary Talks} \label{sec:plenary-talks}
The main seminar material was driven by plenary talks on a variety of topics
ranging, such as trust management, namespace management, privacy, and anonymity.
These talks began with a discussion of threat models and their importance in ICN.
In particular, recall that ICNs attempt to diverge from IP with respect
to the central abstraction of hosts and point-to-point communication between
them. The ICN emphasis on named data and object security instead channel security
is one clear differentiating factor contributing to this parity. To quantify
the degree by which secret is improved (or worsened), threat models are needed.
In general, they must capture a particular design challenge in ICN, such as
infrastructure protection, user-friendly key distribution and trust management
and enforcement, and content protection and access control. And given the wide gap
between IP and ICN, there is a great need for common threat models to use
in the design phrase.

One particular threat revolves around consumer anonymity, which was also
the topic of discussion at the seminar. The subject(s) and contents
of a packet are not the only facets that should be considered with respect to privacy.
Origin and destination details (e.g., geographical location or position within a network
topology), as well as identity information (e.g., consumer identifying information),
can sometimes harm network users. As shown by \cite{}, ICN caching and interest-collapse mechanisms make
ICN itself inherently vulnerable to the possibility for an adversary to locate
consumers. Moreover, an approach similar to the one to violate consumer
(location) privacy, might be used also to detect eavesdropper. Therefore, the
threat model must consider this vulnerability and adversaries capable of exploiting
it successfully.

Another design challenge unique to ICN is how to develop scalable object-based
access control mechanisms. A variety of encryption techniques have been used
in the past to protect access to confidential network data \cite{references}.
Many design approaches, particularly in CCN and NDN, exist well above the network
layer. In contrast, publish-subscribe ICNs such as ENCODERS \cite{} integrate
access control into the network. It uses multi-authority attribute-based encryption
to allow content access to be scoped to selected nodes in the system. Since the
system is completely decentralized, peers serve as brokers that match content from
publishers with interests expressed by subscribers. In order to perform such a match,
an intermediate node must be authorized to see both the relevant content tags and
subscriber interests. One talk at the seminar focused on the observation that
access control policies applied to the metadata (content tags and subscriber interests)
effectively create reachability constraints that are independent from the one defined by
the routing protocols. Consequently, this security-routing interaction must be
treated carefully during policy definition.

The final flavor of plenary talks focused on the future of ICNs. XXX

XXX: crypto algorithms, quantum algorithms, origin, PANINI
